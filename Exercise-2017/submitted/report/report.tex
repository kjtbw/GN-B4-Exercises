\documentclass[fleqn, 14pt]{extarticlej}
\oddsidemargin=-1cm
\usepackage[dvipdfmx]{graphicx}
\usepackage{indentfirst}
\textwidth=18cm
\textheight=23cm
\topmargin=0cm
\headheight=1cm
\headsep=0cm
\footskip=1cm

\def\labelenumi{(\theenumi)}
\def\theenumii{\Alph{enumii}}
\def\theenumiii{(\alph{enumiii})}
\usepackage{comment}
\usepackage{url}

\usepackage{nutils}
\usepackage{jtygm}

\begin{document}

\begin{center}
{\Large {\bf 平成29年度GNグループB4新人研修課題 報告書}}

\end{center}
\begin{flushright}
2017年4月21日\\

乃村研究室 山本 瑛治
\end{flushright}

%かるく課題を書いて,作成できなかった機能がわかるように
\section{概要}
本資料は平成29年度GNグループB4新人研修課題の報告書である.
課題ではSlackBotプログラムを作成した.
%2つの機能の実装を行った.
SlackBotプログラムがもつ機能として
任意の文字列を発言する機能と
Slack以外のサービスと連携する機能
がある.

本資料では,
課題内容,
課題を通して理解できなかった部分,
課題の中で作成できなかった機能,
課題として自主的に作成した機能について述べる.
なお本資料において発言とは
チャットツールであるSlack\cite{Slack}の特定のチャンネル上で発言すること,
または発言そのものを指す.

\section{課題内容}
課題内容は,RubyによるSlackBotプログラムの作成である.
SlackBotプログラムとは
Slackに発言したり,
Slack上の発言を契機に,
何らかの処理を行ったりするプログラムである.
課題として,以下の2つの機能をSlackBotプログラムに実装する.

\begin{enumerate}
\item Slackの特定の発言に対する返信機能

  Slackの発言を取得し,特定の発言に対して返信を行う機能である.
  具体的には,OutgoingWebHooksを利用して発言を取得し,
  取得した発言を基に
  IncomingWebHooksを利用して発言する.
  たとえば,
  SlackBotの取得した発言中に
  \verb|"|「こんにちは」と言って\verb|"|という文字列が存在する場合,
  「」内に含まれる\verb|"|こんにちは\verb|"|という文字列を発言する.
  ここで「」内は\verb|"|こんにちは\verb|"|だけでなく,任意の文字列でも良い.
\item Slack以外のサービスと連携する機能

  WebAPIやWebHooksを経由して他サービスの情報を取得し,
  取得した情報を利用して発言する機能である.
  もしくは,WebAPIやWebHooksを経由してSlackの情報を他サービスに送信する機能である.
  たとえば,Slack上の発言を契機にして他サービスから今日の天気や予定の情報を取得し,
  取得した情報を利用して発言する.
\end{enumerate}
本課題におけるRubyのバージョンは,2.1.5である.

\section{理解できなかった部分}
\begin{enumerate}
%\item オブジェクトという概念
\item Net::HTTP.startメソッドの挙動と引数
%\item 「bundle exec rackup config.ru」コマンドの詳細
\end{enumerate}

\section{作成できなかった機能}
作成できなかった機能を以下に示す.
\begin{enumerate}
\item 設定したOutgoing WebHooks以外からのPOSTを拒否する機能
\item 国名コードを利用して国名を表示する機能
  
  Wikipediaでは記念日の項目中で国名を表示している箇所がある.
  この国名の表示には国を一意に識別できる国コードを利用している.
  今回作成したSlackBotプログラムは国コードと国名の対応表をもっていない.
  そのため国コードから国名を表示できない.
  本機能を実装できれば,Wikipediaと同様に国名を表示できる.
  
  
\end{enumerate}

\section{自主的に作成した機能}
以下の機能を自主的に作成した.
\begin{enumerate}
\item 指定された日付に関する記念日・年中行事を発言する機能
\end{enumerate}


\bibliographystyle{ipsjunsrt}
\bibliography{mybibdata}



 \end{document}
