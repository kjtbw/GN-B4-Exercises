\documentclass[fleqn, 14pt]{extarticlej}
\oddsidemargin=-1cm
\usepackage[dvipdfmx]{graphicx}
\usepackage{indentfirst}
\textwidth=18cm
\textheight=23cm
\topmargin=0cm
\headheight=1cm
\headsep=0cm
\footskip=1cm

\def\labelenumi{(\theenumi)}
\def\theenumii{\Alph{enumii}}
\def\theenumiii{(\alph{enumiii})}
\usepackage{comment}
\usepackage{url}

\usepackage{nutils}
\usepackage{jtygm}

\begin{document}

\begin{center}
{\Large {\bf 平成29年度GNグループB4新人研修課題 報告書}}

\end{center}
\begin{flushright}
2017年4月20日\\

乃村研究室 山本 瑛治
\end{flushright}

\section{概要}
本資料は平成29年度GNグループB4新人研修課題の報告書である.本資料では,課題内容,
理解できなかった部分,作成できなかった機能,自主的に作成した機能について述べる.

\section{課題内容}
課題内容は,RubyによるSlackBotプログラムの作成である.具体的には以下の2つを行う.
\begin{enumerate}
  \item 任意の文字列を特定のチャンネルに発言するプログラムの作成
  \item SlackBotプログラムへの機能の追加
\end{enumerate}
本課題におけるRubyのバージョンは,2.1.5である.

\section{理解できなかった部分}
\begin{enumerate}
%\item オブジェクトという概念
\item Net::HTTP.startメソッドの挙動と引数
%\item 「bundle exec rackup config.ru」コマンドの詳細
\end{enumerate}

\section{作成できなかった機能}
作成できなかった機能を以下に示す.
\begin{enumerate}
\item 設定したOutgoing WebHooks以外からのPOSTを拒否する機能
\item 国名コードを利用して国名を表示する機能
  
  Wikipediaでは記念日の項目中で国名を表示している箇所がある.
  この国名の表示には国を一意に識別できる国コードを利用している.
  今回作成したSlackBotプログラムは国コードと国名の対応表をもっていない.
  そのため国コードから国名を表示できない.
  本機能を実装できれば,Wikipediaと同様に国名を表示できる.
  
  
\end{enumerate}

\section{自主的に作成した機能}
以下の機能を自主的に作成した.
\begin{enumerate}
\item 日付を指定する特定の発言に対し,その日付に関するWikipediaの情報を発言する機能
\end{enumerate}

 \end{document}
